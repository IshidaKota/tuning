\documentclass[12pt,a4paper]{jarticle}
\usepackage[dvipdfmx]{graphicx}
%\usepackage[margin=6truemm]{geometry}
%\usepackage{caption}
\usepackage{url}
\begin{document}

ここで特徴的なことは,前節でも指摘したように水温の成層化が非常に急激に生じていることである.水温成層化の過程を詳しく調べると,4月中旬まで水深方向に変化の少ない水温構造が,4月24日頃からまず表層が急激に高温化し,続いて底層水温も急激に上昇することで全水深にわたって高温化している.さらに,この高温状態が数日続いた後,5月4日頃から表層水温が低温化し,それに引き続き底層水温が5月5日に急激な低温化を示してる.このような急激な成層化や水温の急激な上下動が湾奥部の水温環境の特徴といえる.\\
(略)このように湾奥部においては成層形成過程においても海上風の影響が大きく,風の吹き寄せによる表層暖水の堆積や風向の変化による底層水の浸入といった海水流動により全水深的な水温変動を繰り返しながら成層を強化していく.

感潮域のことを考えると河川を鉛直に等量ずつ入れることは得策ではない。




\begin{thebibliography}{99}

\bibitem{kourei_syakai}平成29年版高齢社会白書(全体版)、内閣府ホームページ\url{http://www8.cao.go.jp/kourei/whitepaper/w-2017/zenbun/29pdf_index.html}

\end{thebibliography}

\end{document}