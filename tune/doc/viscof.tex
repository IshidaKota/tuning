\documentclass[fontsize=12pt,paper=a4]{jlreq}
\usepackage{graphicx}
\usepackage{booktabs}
\graphicspath{{../png/}}
\begin{document}

\section{contents}
\begin{center}
    \begin{tabular}{lrr} \toprule
   label & HPRNU & Hrizontal\_mixing\_coefficient \\ \midrule
   Tokyo211 & 0.01 & 0.01 \\
   Tokyo212 & 0.1 & 0.1 \\
   Tokyo213 & 1.0 & 1.0 \\
   Tokyo214*not shown & 5.0 & 5.0 \\
   Tokyo215 & 10.0 & 10.0 \\ \bottomrule
  \end{tabular}
\end{center}
*kh:鉛直渦動拡散係数(多分)  *viscofh:水平渦動拡散係数(多分)
  
全て千葉1号灯標の値
\begin{figure}[hbtp]
   \centering
   \includegraphics[keepaspectratio,width=140mm]{viscof/khupperTokyo211toTokyo215.png}
   \caption{表層、上が鉛直渦動拡散係数、下が水平渦動拡散係数}
\end{figure}
\begin{figure}[hbtp]
    \centering
    \includegraphics[keepaspectratio,width=140mm]{viscof/khbottomTokyo211toTokyo215.png}
    \caption{底層、渦動拡散係数}
 \end{figure}

 \begin{figure}[hbtp]
    \centering
    \includegraphics[keepaspectratio,width=140mm]{viscof/kmupperTokyo211toTokyo215.png}
    \caption{表層、渦動粘性係数}
 \end{figure}

 \begin{figure}[hbtp]
    \centering
    \includegraphics[keepaspectratio,width=140mm]{viscof/kmbottomTokyo211toTokyo215.png}
    \caption{底層、渦動粘性係数}
 \end{figure}
 



\end{document}